\documentclass{article}

\usepackage{listings}
\usepackage{color}

\renewcommand\lstlistingname{Quelltext}

\lstset{
	language=Haskell,
	basicstyle=\small\sffamily,
	numbers=none,
	numberstyle=\tiny,
	frame=tb,
	tabsize=4,
	columns=fixed,
	showstringspaces=false,
	keepspaces,
	commentstyle=\color{red}
	keywordstyle=\color{blue}
}

\title{Exercice 2.5}
\date{18-05-13}
\author{Patrick Marchand}

\begin{document}
\maketitle
\pagenumbering{gobble}
\newpage
\pagenumbering{arabic}

Ce module est ecrit en haskell litteraire et decrit un petit langage 
avec des Integers, des variables et des lambdas.

\begin{lstlisting}
module Main where
\end{lstlisting}

\section{Grammaire}

\paragraph
L'on definit en premier des types qui decrivent les elements de
notre grammaire.

\begin{lstlisting}
type Var = String

data Exp = Enum Int
  | Evar Var
  | Elet Var Exp Exp
  | Ecall Exp Exp
\end{lstlisting}

\paragraph
Ensuite on rajoute une expression 
pour representer les lambdas,
ceci est une fonction anonyme 
qui prend un nom de variable et une 
expression a laquelle l'appliquer.

\begin{lstlisting}
  | Elambda Var Exp
\end{lstlisting}

\paragraph
La valeur Vprim represente des fonctions qui existent 
deja dans le contexte, pour simplifier
je m'en suis debarasser et j'utilise les lambdas 
pour toute formes de fonctions.

\begin{lstlisting}
data Val = Vnum Int
  | Vlambda (Val -> Val)
\end{lstlisting}

\newpage
\section{Evaluation}

\paragraph
Par apres nous avons besoin d'une fonction
pour rechercher les valeurs des variables
dans un contexte donner.

\subparagraph
Elle trouve une clee dans une liste de paires et 
retourne la valeur associer. J'ai changer l'ordre des arguments 
dans la signature afin d'etre plus consistant avec la fonction eval.

\begin{lstlisting}
elookup :: Eq a => [(a, b)] -> a -> b
elookup ((x1, v1):env) x =
 if x == x1 then v1 else elookup env x
\end{lstlisting}

\paragraph
La fonction principale, qui s'occupe d'evaluer
les expressions et retourner leur valeur, 
en se basant sur le contexte fournie.

\begin{lstlisting}
eval :: [(Var, Val)] -> Exp -> Val
\end{lstlisting}

\begin{enumerate}
\item Il n'y a rien a faire d'autre avec un chiffre que le retourner.
\item Si l'on recoit une variable, l'on retourne sa valeur
\item Une expr let permet de definir des nouvelles variables
\item Les fonctions sont elles meme lier a des variables
    et peuvent etre appeler par une expr call.
\item Dans le cas d'un lambda, nous retournons sa valeur
\end{enumerate}

\begin{lstlisting}
eval env (Enum n) = Vnum n

eval env (Evar x) = elookup env x

eval env (Elet x e1 e2) = 
  let v = eval ((x,v):env) e2 in eval ((x,v):env) e2
  
eval env (Ecall fun actual) = 
  case eval env fun of
    Vlambda f -> f (eval env actual)
    
eval env (Elambda v e) = 
  Vlambda (\value -> eval ((v, value) : env) e)
\end{lstlisting}

Le main permet tout simplement de taire le compilateur,
nous ne pouvons directement utiliser notre evaluateur 
sans definir des instances Show pour nos valeurs, ce qui
est un peu difficile avec la definition de Vlambda.

\begin{lstlisting}
main = do print "done"
\end{lstlisting}

\end{document}